\chapter{Einleitung}

\section{Motivation}

Die Motivation dieses Projektes entstand durch die Vorlesung Systemnahe Programmierung im 4. Semester des Studiengangs zum Bachelor of Science - Applied Computer Science. Im Rahmen der Vorlesung wurden Inhalte zur
Programmiersprache Assembler, sowie Kenntnisse über den 8051 Mikrocomputer und der Entwicklungsumgebung MCU-8051 IDE vermittelt. Um dieses neu erlangte Wissen zu vertiefen sollte in Gruppen von bis zu drei Personen eine Anwendung mit den gelehrten Mitteln realisiert werden.


\section{Aufgabenstellung}

Die vorgegebene Aufgabenstellung war recht trivial formuliert. In Gruppen sollte sich eine Anwendung überlegt werden, welche mithilfe der MCU-8051 IDE realisiert werden kann. Einschränkungen bezüglich des Umfangs der Anwendung gab es keine.

In Folge dessen haben wir uns dafür entschieden eine Anzeigetafel zu entwickeln, welche es ermöglicht zweistellige Punktzahlen anzuzeigen. Ebenso soll es möglich sein diese Punktzahlen manuell in einzelnen Schritten hochzuzählen. Eine Begrenzung der Punktzahlen auf ein bestimmtes Limit wird softwareseitig nicht
vorgenommen und ist somit nur durch die Hardware auf 99 je Seite beschränkt.
